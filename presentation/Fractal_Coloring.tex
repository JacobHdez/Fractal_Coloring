\documentclass[11pt]{beamer}
\usetheme[subsectionpage=progressbar,numbering=none,progressbar=frametitle,block=fill]{metropolis}
\usepackage[utf8]{inputenc}
\usepackage[spanish]{babel}
\usepackage{amsmath}
\usepackage{amsfonts}
\usepackage{amssymb}
\usepackage{graphicx}
\usepackage{dsfont}
\usepackage{commath}
\title{Estimación de curvatura para coloración de sistemas dinámicos en el plano complejo}
\author{Jacobo Hernández Varela}
\date{30 de noviembre de 2019}

\newtheorem{defi1}{{Definición}{\it (Orbita)}}[section]
\newtheorem{defi2}{{Definición}{\it (Orbita truncada)}}[section]
\newtheorem{defi3}{{Definición}{\it (Función de Coloreado)}}[section]
\newtheorem{defi4}{{Definición}{\it (Función paleta)}}[section]

\begin{document}
\maketitle

\section{Conceptos Fundamentales}

\subsection{Órbitas e Iteraciones}
\begin{frame}{Órbitas e Iteraciones}
	\begin{itemize}
		\item La $n$-ésima composición de $f$ es denotado por $f^n(z)$. Esto es,
	\[ f^n(z) = f(f(...f(z)...)) \]
	donde $f$ es aplicado a $z$ $n$ veces.
	
		\item Aplicar la función a la previa composición es llamado iteración.
	\end{itemize}
\end{frame}

\begin{frame}{Órbitas e Iteraciones}
	\begin{defi1}
		Dada una función $f:\mathds{C}\to\mathds{C}$, la orbita de un punto $z\in\mathds{C}$ es el conjunto
		\[ O(z) = \set{z, f(z), f^2(z), ...} \]
	\end{defi1}
\end{frame}

\begin{frame}{Órbitas e Iteraciones}
	\begin{itemize}
		\item Cuando calculamos fractales en la computadora, nos interesa una representación finita de las orbitas. Esto es logrado limitando el maximo numero de elementos que contiene la orbita.
		
		\item Definamos dos contantes $M$ y $N_{max}$ como el \textit{valor de rescate} y \textit{el máximo numero de iteraciones} respectivamente.
	\end{itemize}
\end{frame}

\begin{frame}{Órbitas e Iteraciones}
	\begin{defi2}
		Sean
		\[ O(z) = \set{z, f(z), f^2(z),...} \]
		una orbita, y $M$ y $N_{max}$ constantes dadas. Sea $\overline{N}$ el entero no negativo mas pequeño tal que $\abs{f^{\overline{N}}(z)} > M$, y definimos $N = \min\set{\overline{N}, N_{max}}$.\\
		La orbita truncada $O_T(z)$ es el conjunto 
		\[ O_T(z) = \set{z, f(z), f^2(z),...,f^N(z)} \]
	\end{defi2}
\end{frame}

\subsection{Fractales}
\begin{frame}{Fractales}
	\begin{itemize}
		\item Consideremos una función \[ \fullfunction{f}{\mathds{C}}{\mathds{C}}{z}{z^p + c} \] donde la contante $p\in\mathds{N}$, $p\geq 2$ y la semilla $c'in\mathds{C}$.
		
		\item La función $f(z)$ define \textit{un sistema dinámico}
		\[ z_k = z_{k-1}^{p} + c \]
		
		\item Dependiendo del valor de $c$ y el valor inicial $z_0$ las iteraciones se comportan de manera diferente.
\end{itemize}		
\end{frame}

\subsection{Funciones de coloreado y paleta}
\begin{frame}{Funciones de coloreado y paleta}
	\begin{itemize}
		\item Una \textit{imagen} es representada por una matriz de $m\times n$ por puntos discretos llamados \textit{pixeles}.
		\item Cada pixel en la matriz es asociado a un color $RGB$.
%		\item Sea $z_0$ el punto correspondiente a un pixel en el plano complejo.
\item Para calcular el color de cada pixel en una imagen fractal, la órbita truncada $O_T(z)$ se calcula primero. Luego, la función de color es evaluada
	\end{itemize}
\end{frame}

\begin{frame}{Funciones de coloreado y paleta}
	\begin{defi3}
		Una función de coloreado es una función
		\[ u:\mathds{C}\to\mathds{R} \]
	\end{defi3}
	que mapea a la órbita truncada a los reales.
\end{frame}

\begin{frame}{Funciones de coloreado y paleta}
	\begin{defi4}
		Una función paleta $P:\mathds{R}\to\mathds{R}^3$ mapea el indice del color $I$ al espacio de color RGB. Su dominio es $[0,1]$ y el rango $[0,1]^3$.
	\end{defi4}
\end{frame}

\section{Calculando Imagenes Fractales}
\begin{frame}{Calculando Imagenes Fractales}
\begin{enumerate}
	\item Sea $z_0$ la posición correspondiente del pixel en el plano complejo.
	\item Calcula la órbita truncada iterando la formula $z_n = f(z_{n-1})$ empezando en $z_0$ hasta que
	\begin{itemize}
		\item $\abs{z_n} > M$, o
		\item $n = N_{max}$
	\end{itemize}
	
	\item Usando las funciones de color e indice de color, mapea el resultado de la orbita truncada al valor del indice de color.
	
	\item Determina el color RGB del pixel usando la paleta de colores.
\end{enumerate}

Si $\abs{z_n} > M$, el pixel es un punto exterior. En otro caso, si $n = N_{max}$, el punto es un punto interior.

\end{frame}

\section{Estimación de Curvatura}
\begin{frame}{Estimación de Curvatura}
	\begin{itemize}
		\item La estimación promedio de curvatura fue utilizado originalmente por Damien M. Jones para Ultra Fractal en 1999.
		
		\item El coloreado es basicamente aproximar la curvatura de una curva con putos discretos. 
		\item En este caso, la curvatura esta definida por los puntos de la órbita truncada $O_T(z_0)$.
		\item La aproximación usada es 
		\[ t(Z_n) = \abs{ \arctan \del{\frac{z_n - z_{n-1}}{z_{n-1} - z_{n-2}}} } \]
	\end{itemize}
\end{frame}

\end{document}